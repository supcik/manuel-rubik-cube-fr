\chapter{La première couche}

{
\centering
\RubikFaceUpAll{W}
\RubikFaceRight%
{G}{G}{G}%
{X}{G}{X}%
{X}{X}{X}
\RubikFaceFront%
{O}{O}{O}%
{X}{O}{X}%
{X}{X}{X}
\ShowCube{4cm}{1}{%
	\DrawRubikCubeRU
}
\par
}
\medskip

Pour commencer, nous allons résoudre la couche du haut du cube. Nous choisissons de positionner
le cube avec le centre blanc vers le haut, mais vous pouvez choisir une autre couleur si vous préférez.


\section{La croix}

Pour résoudre la première couche du cube, nous commençons
par faire une «croix» sur la face du haut.

\begin{center}
\RubikFaceUp%
{X}{W}{X}%
{W}{W}{W}%
{X}{W}{X}
\RubikFaceRight%
{X}{G}{X}%
{X}{G}{X}%
{X}{X}{X}
\RubikFaceFront%
{X}{O}{X}%
{X}{O}{X}%
{X}{X}{X}
\ShowCube{2cm}{0.5}{%
	\DrawRubikCubeRU
}
\end{center}

Notez qu'il ne suffit pas de mettre les 4~arêtes blanches sur la face du haut, il faut aussi
que l'autre côté des arêtes corresponde avec la couleur des autres centres (orange et vert dans l'exemple ci-dessus).

Cette première couche peut se résoudre de manière assez intuitive et certains n'auront pas besoin
d'aide. Voici cependant des indications pour ceux qui auraient plus de peine.

Notez que vous pouvez faire pivoter la couche du bas de votre cube tant que vous voulez sans «casser» ce que vous avez déjà fait sur les couches du haut. \rrh{D} ou \rrh{Dp}. Ça nous sera bien utile pour la suite.

\subsection{L'arête à déplacer se trouve sur la dernière couche}
\label{subsec:c1d}

Si l'arête que vous souhaitez déplacer pour faire la croix se trouve sur la couche du bas, vous pouvez tourner cette couche du bas pour l'amener sur la bonne face. Nous aurons alors deux cas possibles:

Soit l'arête à la face blanche vers le bas:

\begin{center}
	\RubikFaceDown%
	{X}{W}{X}%
	{X}{Y}{X}%
	{X}{X}{X}
	\RubikFaceRight%
	{X}{X}{X}%
	{X}{G}{X}%
	{X}{X}{X}
	\RubikFaceFront%
	{X}{X}{X}%
	{X}{O}{X}%
	{X}{O}{X}
	\ShowCube{2cm}{0.5}{%
		\DrawRubikCubeRD
	}
\end{center}

et dans ce cas, il suffit de faire tourner la face avant deux fois: \rrh{F}\rrh{F} (ou \rrh{Fp}\rrh{Fp}).

Ou alors l'arête blanche est sur la face avant:

\begin{center}
	\RubikFaceDown%
	{X}{O}{X}%
	{X}{Y}{X}%
	{X}{X}{X}
	\RubikFaceRight%
	{X}{X}{X}%
	{X}{G}{X}%
	{X}{X}{X}
	\RubikFaceFront%
	{X}{X}{X}%
	{X}{O}{X}%
	{X}{W}{X}
	\ShowCube{2cm}{0.5}{%
		\DrawRubikCubeRD
	}
\end{center}

Dans ce cas, nous faisons remonter l'arête avec les mouvements suivants: \rrh{D}\rrh{Sl}\rrh{Dp}\rrh{Slp}


\subsection{L'arête se trouve sur la couche du milieu}

Si l'arête se trouve sur la couche du milieu, comme ceci:

\begin{center}
	\RubikFaceUp%
	{X}{X}{X}%
	{X}{W}{X}%
	{X}{X}{X}
	\RubikFaceRight%
	{X}{X}{X}%
	{W}{G}{X}%
	{X}{X}{X}
	\RubikFaceFront%
	{X}{X}{X}%
	{X}{O}{O}%
	{X}{X}{X}
	\ShowCube{2cm}{0.5}{%
		\DrawRubikCubeRU
	}
\end{center}

On peut amener cette arête en place tout simplement en tournant la face avant dans le sens antihoraire\footnote{Dans le sens contraire des aiguilles d'une montre}: \rrh{Fp}.

Si l'arête se trouve sur la face du milieu, mais qu'elle est mal positionnée

\begin{center}
	\RubikFaceUp%
	{X}{X}{X}%
	{X}{W}{X}%
	{X}{X}{X}
	\RubikFaceRight%
	{X}{X}{X}%
	{O}{G}{X}%
	{X}{X}{X}
	\RubikFaceFront%
	{X}{X}{X}%
	{X}{O}{W}%
	{X}{X}{X}
	\ShowCube{2cm}{0.5}{%
		\DrawRubikCubeRU
	}
\end{center}

alors on peut déplace cette arête sur la couche du bas: \rrh{Rp}\rrh{Dp}\rrh{R} et on ramène l'arête correctement positionnée sur la couche du haut: \rrh{F}\rrh{F}


Si l'arête se trouve sur la couche du milieu, mais n'est pas sur la bonne face:

\begin{center}
	\RubikFaceUp%
	{X}{X}{X}%
	{X}{W}{X}%
	{X}{X}{X}
	\RubikFaceRight%
	{X}{X}{X}%
	{W}{R}{X}%
	{X}{X}{X}
	\RubikFaceFront%
	{X}{X}{X}%
	{X}{G}{O}%
	{X}{X}{X}
	\ShowCube{2cm}{0.5}{%
		\DrawRubikCubeRU
	}
\end{center}

alors on déplace cette arête sur la couche du bas: \rrh{F}\rrh{Dp}\rrh{Fp} et on applique la règle pour la couche du bas comme expliqué en \ref{subsec:c1d}.


Pour gagner du temps, lorsqu'on déplace une arête sur la couche du bas, on va positionner l'arête de manière à ce que sa face blanche soit vers le bas. C'est le cas pour l'exemple ci-dessus.

Si l'arête de positionnée comme dans le cube ci-dessous:

\begin{center}
	\RubikFaceUp%
	{X}{X}{X}%
	{X}{W}{X}%
	{X}{X}{X}
	\RubikFaceRight%
	{X}{X}{X}%
	{O}{R}{X}%
	{X}{X}{X}
	\RubikFaceFront%
	{X}{X}{X}%
	{X}{G}{W}%
	{X}{X}{X}
	\ShowCube{2cm}{0.5}{%
		\DrawRubikCubeRU
	}
\end{center}

Alors on fait: \rrh{Rp}\rrh{Dp}\rrh{R}

\subsection{L'arête se trouve sur la couche du haut}

Si l'arête se trouve sur la couche du haut, mais que ses couleurs sont inversées, comme dans l'exemple ci-dessous:

\begin{center}
	\RubikFaceUp%
	{X}{X}{X}%
	{X}{W}{X}%
	{X}{O}{X}
	\RubikFaceRight%
	{X}{X}{X}%
	{X}{G}{X}%
	{X}{X}{X}
	\RubikFaceFront%
	{X}{W}{X}%
	{X}{O}{X}%
	{X}{X}{X}
	\ShowCube{2cm}{0.5}{%
		\DrawRubikCubeRU
	}
\end{center}

On peut faire pivoter l'arête avec la séquence suivante: \rrh{F}\rrh{F}\rrh{D}\rrh{Sl}\rrh{Dp}\rrh{Slp}.

Les deux premiers mouvements mettent l'arête sur la dernière couche et la suite est la même séquence que dans la section \ref{subsec:c1d}.

\begin{samepage}
Si l'arête est déjà bien orientée, mais qu'elle n'est pas au bon endroit, comme dans l'exemple ci-dessous:

\begin{center}
	\RubikFaceUp%
	{X}{X}{X}%
	{X}{W}{W}%
	{X}{X}{X}
	\RubikFaceRight%
	{X}{O}{X}%
	{X}{G}{X}%
	{X}{X}{X}
	\RubikFaceFront%
	{X}{X}{X}%
	{X}{O}{X}%
	{X}{X}{X}
	\ShowCube{2cm}{0.5}{%
		\DrawRubikCubeRU
	}
\end{center}
\end{samepage}


Si c'est la première arête que vous mettez en place, vous pouvez simplement tourner la face du haut: \rrh{U}, mais si les autres arêtes sont déjà en place, vous pouvez alors amener l'arête sur la dernière couche: \rrh{Rp}\rrh{Rp}. L'arête se trouve alors sur la dernière couche et on applique la méthode expliquée en \ref{subsec:c1d}.

Si l'arête se trouve sur la dernière couche et qu'elle n'est ni bien orientée, ni bien positionnée:

\begin{center}
	\RubikFaceUp%
	{X}{X}{X}%
	{X}{W}{O}%
	{X}{X}{X}
	\RubikFaceRight%
	{X}{W}{X}%
	{X}{G}{X}%
	{X}{X}{X}
	\RubikFaceFront%
	{X}{X}{X}%
	{X}{O}{X}%
	{X}{X}{X}
	\ShowCube{2cm}{0.5}{%
		\DrawRubikCubeRU
	}
\end{center}

Vous pouvez l'amener sur la couche du bas: \rrh{Rp}\rrh{Rp}\rrh{Dp}\rrh{R}\rrh{R} et ensuite, appliquer la méthode expliquée en \ref{subsec:c1d}.

Notez que dans l'exemple ci-dessus, vous pouvez mettre l'arête en place en continuant avec la simple séquence suivante: \rrh{Rp}\rrh{Fp}

\section{Les coins de la première couche}

Pour terminer la première couche, il ne nous reste plus qu'à mettre les coins en place.

\subsection{Le coin à déplacer se trouve sur la couche du bas}
\label{subsec:c1cd}

Si le coin se trouve sur la dernière couche, il y a trois cas possibles.
Le premier cas est celui où le coin est placé avec le côté blanc vers
la face et il doit monter en diagonale: 

\begin{center}  	
	\RubikFaceRight%
	{X}{G}{X}%
	{X}{G}{X}%
	{X}{X}{X}
	\RubikFaceFront%
	{X}{O}{X}%
	{X}{O}{X}%
	{W}{X}{X}
	\RubikFaceDown%
	{G}{X}{X}%
	{X}{Y}{X}%
	{X}{X}{X}
	
	\ShowCube{2cm}{0.5}{%
	\DrawRubikCubeRD
	\tikzset{>=latex}
	\draw[thick,->] (0.5,0.5) -- (2.5,2.5);
	}
\end{center} 

On résout ce cas avec la séquence suivante:
\rrh{D}\rrh{D}\rrh{F}\rrh{Dp}\rrh{Fp}

Le deuxième cas est celui où le coin est placé avec le côté blanc vers
la face et il doit monter en vertical: 

\begin{center}
	\RubikFaceRight%
	{X}{G}{X}%
	{X}{G}{X}%
	{G}{X}{X}
	\RubikFaceFront%
	{X}{O}{X}%
	{X}{O}{X}%
	{X}{X}{W}
	\RubikFaceDown%
	{X}{X}{O}%
	{X}{Y}{X}%
	{X}{X}{X}
	
	\ShowCube{2cm}{0.5}{%
		\DrawRubikCubeRD
		\tikzset{>=latex}
		\draw[thick,->] (2.5,0.5) -- (2.5,2.5);
	}
\end{center} 

On résout ce cas avec la séquence suivante:
\rrh{Dp}\rrh{Rp}\rrh{D}\rrh{R}


Le troisième cas est celui où le coin est placé avec le côté blanc vers
le bas. On commence par placer le coin à la verticale de la position
vers laquelle on souhaite l'amener:

\begin{center}
	\RubikFaceRight%
	{X}{G}{X}%
	{X}{G}{X}%
	{O}{X}{X}
	\RubikFaceFront%
	{X}{O}{X}%
	{X}{O}{X}%
	{X}{X}{G}
	\RubikFaceDown%
	{X}{X}{W}%
	{X}{Y}{X}%
	{X}{X}{X}
	
	\ShowCube{2cm}{0.5}{%
		\DrawRubikCubeRD
		\tikzset{>=latex}
		\draw[thick,<->] (2.5,0.5) -- (2.5,2.5);
	}
\end{center} 

La séquence suivante permet de faire pivoter le coin et le mettre en bas à gauche: \rrh{Rp}\rrh{D}\rrh{D}\rrh{R}

\begin{samepage}
Le résultat sera:

\begin{center}
	\RubikFaceRight%
	{X}{G}{X}%
	{X}{G}{X}%
	{X}{X}{X}
	\RubikFaceFront%
	{X}{O}{X}%
	{X}{O}{X}%
	{W}{X}{X}
	\RubikFaceDown%
	{G}{X}{X}%
	{X}{Y}{X}%
	{X}{X}{X}
	
	\ShowCube{2cm}{0.5}{%
		\DrawRubikCubeRD
	}
\end{center} 
\end{samepage}
	
Ce qui correspond à un cas connu et nous savons alors comment le faire
venir à sa position finale.

\subsection{Le coin à déplacer se trouve sur la face du haut}

Les coins peuvent déjà se trouver soit sur la couche du haut, mais leur
position n'est peut-être pas la bonne.

Pour déplacer un coin, nous devons commencer par l'amener sur la couche du bas. Prenons l'exemple du cube ci-dessous:

\begin{center}
	\RubikFaceUp%
	{X}{W}{X}%
	{W}{W}{W}%
	{X}{W}{G}
	\RubikFaceRight%
	{O}{R}{X}%
	{X}{R}{X}%
	{X}{X}{X}
	\RubikFaceFront%
	{X}{G}{W}%
	{X}{G}{X}%
	{X}{X}{X}
	\ShowCube{2cm}{0.5}{%
		\DrawRubikCubeRU
	}
\end{center}

Nous avons 4~possibilités pour faire descendre ce coin sur la dernière couche:

\begin{itemize}
	\item \rrh{F}\rrh{D}\rrh{Fp}
	\item \rrh{F}\rrh{Dp}\rrh{Fp}
	\item \rrh{Rp}\rrh{D}\rrh{R}
	\item \rrh{Rp}\rrh{Dp}\rrh{R}		
\end{itemize}

Comme nous avons vu en \ref{subsec:c1cd}, c'est plus rapide si la face blanche du coin \textbf{n'est pas} dirigée vers le bas.

\begin{samepage}
Dans le cas ci-dessus, nous choisirons plutôt les variantes 1 et 3 qui se terminent avec les cubes suivants:

\begin{center}	
	\RubikFaceRight%
	{X}{R}{X}%
	{X}{R}{X}%
	{X}{X}{W}
	\RubikFaceFront%
	{X}{G}{X}%
	{X}{G}{X}%
	{X}{X}{X}
	\RubikFaceDown%
	{X}{X}{X}%
	{X}{Y}{X}%
	{X}{X}{O}
	\ShowCube{2cm}{0.5}{%
		\DrawRubikCubeRD
	}%
	\hspace*{1cm} 	
	\RubikFaceRight%
	{X}{R}{X}%
	{X}{R}{X}%
	{G}{X}{X}
	\RubikFaceFront%
	{X}{G}{X}%
	{X}{G}{X}%
	{X}{X}{W}
	\RubikFaceDown%
	{X}{X}{O}%
	{X}{Y}{X}%
	{X}{X}{X}
	\ShowCube{2cm}{0.5}{%
		\DrawRubikCubeRD
	}
\end{center} 
\end{samepage}
	
On pourrait bien aussi avoir le cas où le coin est bien placé, mais mal orienté:

\begin{center}
	\RubikFaceUp%
	{X}{W}{X}%
	{W}{W}{W}%
	{X}{W}{O}
	\RubikFaceRight%
	{W}{G}{X}%
	{X}{G}{X}%
	{X}{X}{X}
	\RubikFaceFront%
	{X}{O}{G}%
	{X}{O}{X}%
	{X}{X}{X}
	\ShowCube{2cm}{0.5}{%
		\DrawRubikCubeRU
	}
\end{center}

Dans ce cas également nous l'amènerons sur la couche du bas, par exemple avec \rrh{Rp}\rrh{Dp}\rrh{R}.
 
Ce qui donne:
 
\begin{center}
    \RubikFaceRight%
 	{X}{G}{X}%
 	{X}{G}{X}%
 	{X}{X}{X}
 	\RubikFaceFront%
 	{X}{O}{X}%
 	{X}{O}{X}%
 	{W}{X}{X}
 	\RubikFaceDown%
 	{G}{X}{X}%
 	{X}{Y}{X}%
 	{X}{X}{X}
 	\ShowCube{2cm}{0.5}{%
 		\DrawRubikCubeRD
 	}
\end{center} 


Ce cas est connu est nous savons comment faire monter ce coin en diagonale.

Voilà, vous avez maintenant toutes les informations pour résoudre la première
couche du cube. Entraînez-vous plusieurs fois à faire cette couche.
