\chapter{La couche du milieu}

{
	\centering
	\RubikFaceUpAll{W}
	\RubikFaceRight%
	{G}{G}{G}%
	{G}{G}{G}%
	{X}{X}{X}
	\RubikFaceFront%
	{O}{O}{O}%
	{O}{O}{O}%
	{X}{X}{X}
	\ShowCube{4cm}{1}{%
		\DrawRubikCubeRU
	}
	\par
}

L'histoire que nous utilisons pour mémoriser les mouvements est connue sous \og l'histoire du Belge\fg{}. Mais comme nous aimons bien les Belges et que nous ne voulons pas de problème avec eux, nous appellerons plutôt cette histoire \og l'histoire du distrait\fg{}.

\begin{itemize}
	\item Il doit aller à droite, mais il se trompe et part à gauche: \rrh{Dp}.
	\item Ses amis viennent le chercher: \rrh{Rp}.
	\item Il se dirige alors dans le bon sens: \rrh{D}.
	\item Ses amis rentrent chez eux: \rrh{R}.
	\item Emporté par son élan, il continue trop loin: \rrh{D}.
	\item Il va tellement vite qu'il emporte la face avant: \rrh{F}.
	\item Il remarque enfin son erreur et retourne sur ses pas: \rrh{Dp}.
	\item La face avant peut se remettre en place: \rrh{Fp}.
	
\end{itemize}
