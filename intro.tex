\chapter{Introduction}

Le Rubik's cube a été inventé en 1974 par le sculpteur et professeur
d'architecture hongrois \emph{Ernő Rubik}. Pour la petite histoire, il a
fallu plus d'un mois à Ernő Rubik pour résoudre sa propre invention! Ce
cube était très populaire dans les années~80, et aujourd'hui encore, il
reste un objet très apprécié par tous ceux qui s'intéressent aux
sciences, aux mathématiques, ou à la technologie.

Les règles du jeu sont extrêmement simples: il suffit de faire pivoter
les parties du cube de manière à rassembler toutes les pastilles de la
même couleur sur la même face:

\begin{center}
	\RubikCubeSolved
	\ShowCube{2cm}{0.5}{%
		\DrawRubikCubeRU
	}
\end{center}

Mais la simplicité s'arrête là. En effet, il y a plus de
43~trillions\footnote{Pour être précis, il y a
43\,252\,003\,274\,489\,856\,000 configurations possibles.}
configurations possibles du cube et il est très difficile de prévoir
quels mouvements seront nécessaires pour résoudre un cube bien
«mélangé».

Des chercheurs ont démontré\cite{god20} qu'on pouvait résoudre n'importe
quel cube avec un maximum de 20~mouvements. La méthode \emph{couche par
couche} présentée dans ce livre nécessite beaucoup plus que
20~mouvements, mais elle a l'avantage d'être bien adaptée aux débutants.
Si, plus tard, vous souhaitez battre des records de vitesse, vous devrez
apprendre d'autres méthodes; plus rapides, mais aussi plus difficiles à
mémoriser.
 
Nous commencerons par positionner toutes les pièces de la couche du
haut, ensuite nous positionnerons les pièces de la couche du milieu et
nous terminerons par les pièces de la dernière couche. Prenez le temps
de bien exercer chaque couche avant de passer à la suivante. Vous
n'apprendrez pas plus vite en brûlant les étapes. Si vous utilisez ce
livre dans le cadre d'une série d'ateliers, vous pouvez très bien faire
trois séances de une heure chacune. Vous étudierez alors une couche par
séance et vous aurez du temps pour vous exercer entre les séances. 

