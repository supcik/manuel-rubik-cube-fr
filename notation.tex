\chapter{Notation}
La littérature sur le Rubik's cube utilise souvent une notation spécifique pour décrire les mouvements. La table de la page suivante décrit la notation la plus souvent utilisée.

Les lettres correspondent à la description de la face (en anglais): \textbf{R}ight (droit), \textbf{L}eft (gauche), \textbf{U}p (dessus), \textbf{D}own (dessous), \textbf{F}ront (avant), \textbf{B}ack (arrière).
Si la lettre est seule, il faut tourner dans le sens des aiguilles d'une montre.
Si la lettre est suivie d'une apostrophe (') alors il faut tourner dans le sens \textbf{contraire} des aiguilles d'une montre. Si la lettre est suivie du chiffre 2, il faut répéter deux fois le mouvement (ce qui revient à lui faire faire un demi-tour, et le sens de rotation est donc égal).

Cette notation est également utilisée dans les concours pour indiquer comment \emph{mélanger} un cube\cite{scramble}. En partant du cube terminé, faites la séquence suivante:

\textbf{L2 U' B2 U2 B' -- U B U' L' F' -- L2 B D R D2 -- B2 F2 R' U L2 -- F2 D2 R B F}

Le résultat sera:

\begin{center}
	\RubikFaceRight%
	{R}{R}{W}%
	{G}{G}{R}%
	{O}{R}{O}
	\RubikFaceFront%
	{O}{O}{G}%
	{O}{O}{O}%
	{B}{Y}{W}
	\RubikFaceUp%
	{G}{B}{B}%
	{G}{W}{W}%
	{B}{Y}{W}
	\ShowCube{2cm}{0.5}{%
		\DrawRubikCubeRU
	}
\end{center} 

\setlength{\tabcolsep}{8pt}
\begin{center}
\begin{tabular}{ccccc}
{\Huge\textbf{R}} &
\RubikFaceRight%
{X}{X}{X}%
{X}{X}{X}%
{X}{X}{X}
\RubikFaceFront%
{X}{X}{B}%
{X}{X}{B}%
{X}{X}{B}
\RubikFaceUp%
{X}{X}{B}%
{X}{X}{B}%
{X}{X}{B}
\ShowCube{2cm}{0.5}{%
	\DrawRubikCubeRU
	\tikzset{>=latex}
	\draw[red, thick,->] (2.5,0.5) -- (2.5,3) -- (2.5+5/6,3+5/6);
}
& \hspace*{5mm} &
{\Huge\textbf{R'}} &
\RubikFaceRight%
{X}{X}{X}%
{X}{X}{X}%
{X}{X}{X}
\RubikFaceFront%
{X}{X}{B}%
{X}{X}{B}%
{X}{X}{B}
\RubikFaceUp%
{X}{X}{B}%
{X}{X}{B}%
{X}{X}{B}
\ShowCube{2cm}{0.5}{%
	\DrawRubikCubeRU
	\tikzset{>=latex}
	\draw[red, thick,<-] (2.5,0.5) -- (2.5,3) -- (2.5+5/6,3+5/6);
}
\\
\noalign{\medskip}
{\Huge\textbf{L}} &

\RubikFaceRight%
{X}{X}{X}%
{X}{X}{X}%
{X}{X}{X}
\RubikFaceFront%
{B}{X}{X}%
{B}{X}{X}%
{B}{X}{X}
\RubikFaceUp%
{B}{X}{X}%
{B}{X}{X}%
{B}{X}{X}
\ShowCube{2cm}{0.5}{%
	\DrawRubikCubeRU
	\tikzset{>=latex}
	\draw[red, thick,<-] (0.5,0.5) -- (0.5,3) -- (0.5+5/6,3+5/6);
}
& \hspace*{5mm} &
{\Huge\textbf{L'}} &
\RubikFaceRight%
{X}{X}{X}%
{X}{X}{X}%
{X}{X}{X}
\RubikFaceFront%
{B}{X}{X}%
{B}{X}{X}%
{B}{X}{X}
\RubikFaceUp%
{B}{X}{X}%
{B}{X}{X}%
{B}{X}{X}
\ShowCube{2cm}{0.5}{%
	\DrawRubikCubeRU
	\tikzset{>=latex}
	\draw[red, thick,->] (0.5,0.5) -- (0.5,3) -- (0.5+5/6,3+5/6);
}
\\
\noalign{\medskip}
{\Huge\textbf{U}} &
\RubikFaceRight%
{B}{B}{B}%
{X}{X}{X}%
{X}{X}{X}
\RubikFaceFront%
{B}{B}{B}%
{X}{X}{X}%
{X}{X}{X}
\RubikFaceUp%
{X}{X}{X}%
{X}{X}{X}%
{X}{X}{X}
\ShowCube{2cm}{0.5}{%
	\DrawRubikCubeRU
	\tikzset{>=latex}
	\draw[red, thick,<-] (0.5,2.5) -- (3,2.5) -- (3+5/6,2.5+5/6);
}
& \hspace*{5mm} &
{\Huge\textbf{U'}} &
\RubikFaceRight%
{B}{B}{B}%
{X}{X}{X}%
{X}{X}{X}
\RubikFaceFront%
{B}{B}{B}%
{X}{X}{X}%
{X}{X}{X}
\RubikFaceUp%
{X}{X}{X}%
{X}{X}{X}%
{X}{X}{X}
\ShowCube{2cm}{0.5}{%
	\DrawRubikCubeRU
	\tikzset{>=latex}
	\draw[red, thick,->] (0.5,2.5) -- (3,2.5) -- (3+5/6,2.5+5/6);
}
\\
\noalign{\medskip}
{\Huge\textbf{D}} &
\RubikFaceRight%
{X}{X}{X}%
{X}{X}{X}%
{B}{B}{B}
\RubikFaceFront%
{X}{X}{X}%
{X}{X}{X}%
{B}{B}{B}
\RubikFaceUp%
{X}{X}{X}%
{X}{X}{X}%
{X}{X}{X}
\ShowCube{2cm}{0.5}{%
	\DrawRubikCubeRU
	\tikzset{>=latex}
	\draw[red, thick,->] (0.5,0.5) -- (3,0.5) -- (3+5/6,0.5+5/6);
}
& \hspace*{5mm} &
{\Huge\textbf{D'}} &
\RubikFaceRight%
{X}{X}{X}%
{X}{X}{X}%
{B}{B}{B}
\RubikFaceFront%
{X}{X}{X}%
{X}{X}{X}%
{B}{B}{B}
\RubikFaceUp%
{X}{X}{X}%
{X}{X}{X}%
{X}{X}{X}
\ShowCube{2cm}{0.5}{%
	\DrawRubikCubeRU
	\tikzset{>=latex}
	\draw[red, thick,<-] (0.5,0.5) -- (3,0.5) -- (3+5/6,0.5+5/6);
}
\\
\noalign{\medskip}
{\Huge\textbf{F}} &
\RubikFaceRight%
{B}{X}{X}%
{B}{X}{X}%
{B}{X}{X}
\RubikFaceFront%
{X}{X}{X}%
{X}{X}{X}%
{X}{X}{X}
\RubikFaceUp%
{X}{X}{X}%
{X}{X}{X}%
{B}{B}{B}
\ShowCube{2cm}{0.5}{%
	\DrawRubikCubeRU
	\tikzset{>=latex}
	\draw[red, thick,->] (0.5+1/6,3+1/6) -- (3+1/6,3+1/6) -- (3+1/6,0.5+1/6);
}
& \hspace*{5mm} &
{\Huge\textbf{F'}} &
\RubikFaceRight%
{B}{X}{X}%
{B}{X}{X}%
{B}{X}{X}
\RubikFaceFront%
{X}{X}{X}%
{X}{X}{X}%
{X}{X}{X}
\RubikFaceUp%
{X}{X}{X}%
{X}{X}{X}%
{B}{B}{B}
\ShowCube{2cm}{0.5}{%
	\DrawRubikCubeRU
	\tikzset{>=latex}
	\draw[red, thick,<-] (0.5+1/6,3+1/6) -- (3+1/6,3+1/6) -- (3+1/6,0.5+1/6);
}
\\
\noalign{\medskip}
{\Huge\textbf{B}} &
\RubikFaceRight%
{X}{X}{B}%
{X}{X}{B}%
{X}{X}{B}
\RubikFaceFront%
{X}{X}{X}%
{X}{X}{X}%
{X}{X}{X}
\RubikFaceUp%
{B}{B}{B}%
{X}{X}{X}%
{X}{X}{X}
\ShowCube{2cm}{0.5}{%
	\DrawRubikCubeRU
	\tikzset{>=latex}
	\draw[red, thick,<-] (0.5+5/6,3+5/6) -- (3.5+2/6,3+5/6) -- (3.5+2/6,1+1/6);
}
& \hspace*{5mm} &
{\Huge\textbf{B'}} &
\RubikFaceRight%
{X}{X}{B}%
{X}{X}{B}%
{X}{X}{B}
\RubikFaceFront%
{X}{X}{X}%
{X}{X}{X}%
{X}{X}{X}
\RubikFaceUp%
{B}{B}{B}%
{X}{X}{X}%
{X}{X}{X}
\ShowCube{2cm}{0.5}{%
	\DrawRubikCubeRU
	\tikzset{>=latex}
	\draw[red, thick,->] (0.5+5/6,3+5/6) -- (3.5+2/6,3+5/6) -- (3.5+2/6,1+1/6);
}
\end{tabular}
\end{center}