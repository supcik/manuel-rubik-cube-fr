\chapter{Les différentes parties du cube}

Commençons par étudier les différentes parties du Rubik's cube.

\section{Les centres}

Le cube se compose de 6 \textbf{centres} qui sont toujours placés de la même manière:

\begin{center}
	\RubikFaceUp%
	{X}{X}{X}%
	{X}{W}{X}%
	{X}{X}{X}
	\RubikFaceRight%
	{X}{X}{X}%
	{X}{G}{X}%
	{X}{X}{X}
	\RubikFaceFront%
	{X}{X}{X}%
	{X}{O}{X}%
	{X}{X}{X}
	\ShowCube{2cm}{0.5}{%
		\DrawRubikCubeRU
	}
\end{center}

Les centres sont identifiés par une \textbf{pastille}\footnote{La plupart des cubes du commerce ont des auto-collants pour identifier les couleurs.} Les
couleurs du cube original sont blanc, rouge, bleu, orange, vert et jaune.
Si votre cube a d'autres couleurs, ce n'est pas grave, la méthode reste la
même.

Les centres restent toujours à la même place; vous pouvez faire tous les mouvements que vous voulez, vous ne changerez jamais la position des centres.

\section{Les arêtes}
\begin{samepage}
Le cube se compose également de 12 \textbf{arêtes}:

\begin{center}
	\RubikFaceUp%
	{X}{W}{X}%
	{W}{X}{W}%
	{X}{W}{X}
	\RubikFaceRight%
	{X}{G}{X}%
	{G}{X}{G}%
	{X}{G}{X}
	\RubikFaceFront%
	{X}{O}{X}%
	{O}{X}{O}%
	{X}{O}{X}
	\ShowCube{2cm}{0.5}{%
		\DrawRubikCubeRU
	}
\end{center}
\end{samepage}

Les arêtes sont les pièces placées entre les centres et elles ont toutes 2 pastilles de couleur différentes.

\section{Les sommets}

Pour terminer le cube a 8 \textbf{sommets}:

\begin{center}
	\RubikFaceUp%
	{W}{X}{W}%
	{X}{X}{X}%
	{W}{X}{W}
	\RubikFaceRight%
	{G}{X}{G}%
	{X}{X}{X}%
	{G}{X}{G}
	\RubikFaceFront%
	{O}{X}{O}%
	{X}{X}{X}%
	{O}{X}{O}
	\ShowCube{2cm}{0.5}{%
		\DrawRubikCubeRU
	}
\end{center}

Chaque sommet a 3 pastilles de couleur différentes.

Il reste encore une pièce que nous ne voyons pas et qui est au milieu du cube. Si on additionne tous les types de pièces, on a $6 + 12 + 8 + 1 = 27$, ce qui correspond bien à ce que nous attendions avec un cube de $3 \times 3 \times 3$.

\section{Les couches}
\begin{samepage}
Une \textbf{couche} peut être comparée à un «étage» du cube. Le Rubik's cube est composé de trois couches:

\begin{center}
	\RubikFaceUp%
	{W}{W}{W}%
	{W}{W}{W}%
	{W}{W}{W}
	\RubikFaceRight%
	{G}{G}{G}%
	{X}{X}{X}%
	{X}{X}{X}
	\RubikFaceFront%
	{O}{O}{O}%
	{X}{X}{X}%
	{X}{X}{X}
	\ShowCube{2cm}{0.5}{%
		\DrawRubikCubeRU
		\node[anchor=center] at(1.5,2.5) {1};
	}
	\hspace*{5mm}
	\RubikFaceUp%
	{X}{X}{X}%
	{X}{X}{X}%
	{X}{X}{X}
	\RubikFaceRight%
	{X}{X}{X}%
	{G}{G}{G}%
	{X}{X}{X}
	\RubikFaceFront%
	{X}{X}{X}%
	{O}{O}{O}%
	{X}{X}{X}
	\ShowCube{2cm}{0.5}{%
		\DrawRubikCubeRU
		\node[anchor=center] at(1.5,1.5) {2};
	}
	\hspace*{5mm}
	\RubikFaceUp%
	{X}{X}{X}%
	{X}{X}{X}%
	{X}{X}{X}
	\RubikFaceRight%
	{X}{X}{X}%
	{X}{X}{X}%
	{G}{G}{G}
	\RubikFaceFront%
	{X}{X}{X}%
	{X}{X}{X}%
	{O}{O}{O}
	\ShowCube{2cm}{0.5}{%
		\DrawRubikCubeRU
		\node[anchor=center] at(1.5,0.5) {3};
	}
\end{center}
\end{samepage}
	
Avec la méthode proposée dans ce livre, vous commencerez par faire la première couche (parfois aussi appelée couche du haut), puis vous passerez à la deuxième couche (ou couche du milieu) et vous terminerez avec la troisième couche (ou couche du bas).

\section{Les faces}

\begin{center}
	\RubikFaceUp%
	{W}{W}{W}%
	{W}{W}{W}%
	{W}{W}{W}
	\RubikFaceRight%
	{X}{X}{X}%
	{X}{X}{X}%
	{X}{X}{X}
	\RubikFaceFront%
	{X}{X}{X}%
	{X}{X}{X}%
	{X}{X}{X}
	\ShowCube{2cm}{0.5}{%
		\DrawRubikCubeRU
	}
	\hspace*{5mm}
	\RubikFaceUp%
	{X}{X}{X}%
	{X}{X}{X}%
	{X}{X}{X}
	\RubikFaceRight%
	{G}{G}{G}%
	{G}{G}{G}%
	{G}{G}{G}
	\RubikFaceFront%
	{X}{X}{X}%
	{X}{X}{X}%
	{X}{X}{X}
	\ShowCube{2cm}{0.5}{%
		\DrawRubikCubeRU
	}
	\hspace*{5mm}
	\RubikFaceUp%
	{X}{X}{X}%
	{X}{X}{X}%
	{X}{X}{X}
	\RubikFaceRight%
	{X}{X}{X}%
	{X}{X}{X}%
	{X}{X}{X}
	\RubikFaceFront%
	{O}{O}{O}%
	{O}{O}{O}%
	{O}{O}{O}
	\ShowCube{2cm}{0.5}{%
		\DrawRubikCubeRU
	}
\end{center}

