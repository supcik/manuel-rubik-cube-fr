\chapter{La dernière couche}


{
	\centering
	\RubikFaceUpAll{W}
	\RubikFaceRight%
	{G}{G}{G}%
	{G}{G}{G}%
	{G}{G}{G}
	\RubikFaceFront%
	{O}{O}{O}%
	{O}{O}{O}%
	{O}{O}{O}
	\ShowCube{4cm}{1}{%
		\DrawRubikCubeRU
	}
	\par
}

Pour la dernière couche, nous retournons le cube. Si nous avons commencé avec la face blanche, ça veut dire que nous aurons la face jaune vers le haut. 

\section{La Croix}

Comme pour la première couche, nous commençons par réaliser une croix jaune.

\subsection{Une ligne sur la face}

\begin{samepage}
S’il y a une ligne jaune sur le cube, placez la ligne horizontalement comme sur la figure ci-dessous:

\begin{center}	
	\RubikFaceUp%
	{X}{X}{X}%
	{Y}{Y}{Y}%
	{X}{X}{X}
	\RubikFaceRight%
	{X}{X}{X}%
	{X}{B}{X}%
	{X}{X}{X}
	\RubikFaceFront%
	{X}{X}{X}%
	{X}{O}{X}%
	{X}{X}{X}
	\ShowCube{2cm}{0.5}{%
		\DrawRubikCubeRU
	}
\end{center}
\end{samepage}
	
Pour réaliser la croix à partir de cette position, faites la séquence de mouvements suivants:

\rrh{F}\rrh{R}\rrh{U}\rrh{Rp}\rrh{Up}\rrh{Fp}

Pour mémoriser cette suite de mouvements, vous pouvez inventer votre propre histoire. En voici une qui peut vous aider à trouver l'inspiration:

\begin{itemize}
	\item Il va dans le futur en fusée: \rrh{F}.
	\item Les extra-terrestres montent le voir: \rrh{R}.
	\item Il s'enfuit hors de la fusée: \rrh{U}.
	\item Les extra-terrestres rentrent chez eux: \rrh{Rp}.
	\item Il rentre dans la fusée: \rrh{Up}.
	\item Il revient dans le présent: \rrh{Fp}.
\end{itemize}

\subsection{Un «L» sur la face}
\begin{samepage}
Si au lieu d'une ligne, vous avez un «L» inversé sur la face:

\begin{center}	
	\RubikFaceUp%
	{X}{Y}{X}%
	{Y}{Y}{X}%
	{X}{X}{X}
	\RubikFaceRight%
	{X}{X}{X}%
	{X}{B}{X}%
	{X}{X}{X}
	\RubikFaceFront%
	{X}{X}{X}%
	{X}{O}{X}%
	{X}{X}{X}
	\ShowCube{2cm}{0.5}{%
		\DrawRubikCubeRU
	}
\end{center}
\end{samepage}
	
Vous pouvez obtenir une croix avec la séquence suivante:

\rrh{F}\rrh{U}\rrh{R}\rrh{Up}\rrh{Rp}\rrh{Fp}

Voici une histoire pour vous aider à mémoriser cette séquence.

\begin{itemize}
	\item Il va dans le futur: \rrh{F}.
	\item Il ouvre une boîte: \rrh{U}.
	\item Il en sort un billet: \rrh{R}.
	\item Il referme la boîte: \rrh{Up}.
	\item Il met le billet sans sa poche: \rrh{Rp}.
	\item Il revient dans le présent: \rrh{Fp}.
\end{itemize}

Vous pouvez bien évidemment inventer votre propre histoire si vous voulez.

\subsection{Autre configuration}

Si vous n'avez ni ligne ni «L» inversé, vous pouvez faire l'une ou l'autre des méthodes ci-dessus pour obtenir une configuration connue.

\section{Positionnement des arêtes}

Après avoir fait la croix, il est très probable que les arêtes ne soient pas bien positionnées. La séquence que nous allons voir permet de faire \textit{tourner} les trois arêtes comme dans la figure ci-dessous. Observez bien votre cube, positionnez la dernière couche pour faire correspondre l'arête avec le centre devant vous:

\begin{center}	
	\RubikFaceUp%
	{X}{Y}{X}%
	{Y}{Y}{Y}%
	{X}{Y}{X}
	\RubikFaceRight%
	{X}{G}{X}%
	{X}{B}{X}%
	{X}{X}{X}
	\RubikFaceFront%
	{X}{O}{X}%
	{X}{O}{X}%
	{X}{X}{X}
	\ShowCube{2cm}{0.5}{%
		\DrawRubikCubeRU
	}
\end{center}

Regardez si la rotation des arêtes comme montrée sur la figure suivante permet de bien positionner les arêtes. 
 

\begin{center}	
\ShowCube{2.1cm}{0.7}{%
	\DrawRubikLayerFace
	{X}{Y}{X}
	{Y}{Y}{Y}
	{X}{Y}{X}
	\tikzset{>=latex}
	\draw[thick,->] (2.5,1.5) -- (0.5,1.5);	
	\draw[thick,->] (0.5,1.5) -- (1.5,2.5);	
	\draw[thick,->] (1.5,2.5) -- (2.5,1.5);	
}
\end{center}

Si la couleur que vous avez choisie ne «fonctionne» pas, essayez avec une autre couleur. Si aucune couleur ne fonctionne, choisissez une couleur au hasard.

Pour faire tourner les arêtes comme indiqué sur la figure ci-dessus, nous utilisons \textbf{l'histoire de la chaise}:

\begin{itemize}
	\item Il se lève: \rrh{R}.
	\item Il part très très loin: \rrh{U}\rrh{U}.
	\item Sa chaise tombe: \rrh{Rp}.
	\item Il revient sur ses pas: \rrh{Up}.
	\item Il relève sa chaise: \rrh{R}.
	\item Il revient encore un peu plus sur ses pas: \rrh{Up}.	
	\item Il s'assied: \rrh{Rp}.
\end{itemize}

\section{Positionnement des sommets}

Une fois la croix terminée, nous allons positionner les 4 derniers sommets. Le but pour l'instant est de mettre les sommets au bon endroit, sans s'inquiéter de leurs orientations. Dans la figure ci-dessous, le sommet jaune-orange-bleu est bien placé.

\begin{center}	
	\RubikFaceUp%
	{X}{Y}{X}%
	{Y}{Y}{Y}%
	{X}{Y}{B}
	\RubikFaceRight%
	{O}{B}{X}%
	{X}{B}{X}%
	{X}{X}{X}
	\RubikFaceFront%
	{X}{O}{Y}%
	{X}{O}{X}%
	{X}{X}{X}
	\ShowCube{2cm}{0.5}{%
		\DrawRubikCubeRU
	}
\end{center}

Cherchez un sommet qui est déjà bien placé et mettez-le en haut à droite comme sur la figure ci-dessous. Nous faisons tourner les trois autres sommets avec l'histoire \textbf{des amis}:

\begin{itemize}
	\item Ses amis de gauche montent le voir: \rrh{Lp}.
	\item Il va les saluer:  \rrh{U}.
	\item Ses amis de droite montent le voir: \rrh{R}.
	\item Il va les saluer: \rrh{Up}.
	\item Ses amis de gauche se sentent seuls et descendent chez eux: \rrh{L}.
	\item Il va leur dire au revoir: \rrh{U}.
	\item Ses amis de droite se sentent seuls et descendent chez eux: \rrh{Rp}.
	\item Il va leur dire au revoir: \rrh{Up}.
\end{itemize}

Répétez cette séquence jusqu'à ce que les sommets soient tous bien placés.

\section{Orientation des sommets}

Pour terminer le cube, nous devons encore orienter les sommets. Avec la dernière séquence, nous pouvons faire pivoter les sommets comme indiqué sur la figure ci-dessous:

\begin{center}	
	\RubikFaceUp%
	{X}{Y}{B}%
	{Y}{Y}{Y}%
	{X}{Y}{B}
	\RubikFaceRight%
	{Y}{B}{Y}%
	{X}{B}{X}%
	{X}{X}{X}
	\RubikFaceFront%
	{X}{O}{O}%
	{X}{O}{X}%
	{X}{X}{X}
	\ShowCube{2cm}{0.5}{%
		\DrawRubikCubeRU
	\tikzset{>=latex}
	\draw[->] (3 + 1/6,2.5+0/6) -- (3 + 1/6,2.5+3/6);
	\draw[white,->] (3 + 0/6,2.5+4/6) -- (2.5 + 0/6,2.5+4/6);
	
	\draw[->] (3.5 + 2/6,3.0+1/6) -- (3.5 + 2/6,3.0+4/6);
	\draw[white,->] (3.5 + 1/6,3.0+5/6) -- (3.0 + 1/6,3.0+5/6);	
}
\end{center}

La séquence semble compliquée, mais c'est deux fois l'histoire de la chaise. Une première fois à droite:

\rrh{R}\rrh{U}\rrh{U}\rrh{Rp}\rrh{Up}\rrh{R}\rrh{Up}\rrh{Rp}

\begin{itemize}
	\item Il se lève: \rrh{R}.
	\item Il part très très loin: \rrh{U}\rrh{U}.
	\item Sa chaise tombe: \rrh{Rp}.
	\item Il revient sur ses pas: \rrh{Up}.
	\item Il relève sa chaise: \rrh{R}.
	\item Il revient encore un peu plus sur ses pas: \rrh{Up}.	
	\item Il s'assied: \rrh{Rp}.
\end{itemize}

Et une deuxième fois à gauche:

\rrh{Lp}\rrh{Up}\rrh{Up}\rrh{L}\rrh{U}\rrh{Lp}\rrh{U}\rrh{L}

\begin{itemize}
	\item Il se lève: \rrh{Lp}.
	\item Il part très très loin: \rrh{Up}\rrh{Up}.
	\item Sa chaise tombe: \rrh{L}.
	\item Il revient sur ses pas: \rrh{U}.
	\item Il relève sa chaise: \rrh{Lp}.
	\item Il revient encore un peu plus sur ses pas: \rrh{U}.	
	\item Il s'assied: \rrh{L}.
\end{itemize}

Répétez cette séquence tant que le cube n'est pas terminé.
